% Options for packages loaded elsewhere
% Options for packages loaded elsewhere
\PassOptionsToPackage{unicode}{hyperref}
\PassOptionsToPackage{hyphens}{url}
\PassOptionsToPackage{dvipsnames,svgnames,x11names}{xcolor}
%
\documentclass[
  letterpaper,
  DIV=11,
  numbers=noendperiod]{scrreprt}
\usepackage{xcolor}
\usepackage{amsmath,amssymb}
\setcounter{secnumdepth}{5}
\usepackage{iftex}
\ifPDFTeX
  \usepackage[T1]{fontenc}
  \usepackage[utf8]{inputenc}
  \usepackage{textcomp} % provide euro and other symbols
\else % if luatex or xetex
  \usepackage{unicode-math} % this also loads fontspec
  \defaultfontfeatures{Scale=MatchLowercase}
  \defaultfontfeatures[\rmfamily]{Ligatures=TeX,Scale=1}
\fi
\usepackage{lmodern}
\ifPDFTeX\else
  % xetex/luatex font selection
\fi
% Use upquote if available, for straight quotes in verbatim environments
\IfFileExists{upquote.sty}{\usepackage{upquote}}{}
\IfFileExists{microtype.sty}{% use microtype if available
  \usepackage[]{microtype}
  \UseMicrotypeSet[protrusion]{basicmath} % disable protrusion for tt fonts
}{}
\makeatletter
\@ifundefined{KOMAClassName}{% if non-KOMA class
  \IfFileExists{parskip.sty}{%
    \usepackage{parskip}
  }{% else
    \setlength{\parindent}{0pt}
    \setlength{\parskip}{6pt plus 2pt minus 1pt}}
}{% if KOMA class
  \KOMAoptions{parskip=half}}
\makeatother
% Make \paragraph and \subparagraph free-standing
\makeatletter
\ifx\paragraph\undefined\else
  \let\oldparagraph\paragraph
  \renewcommand{\paragraph}{
    \@ifstar
      \xxxParagraphStar
      \xxxParagraphNoStar
  }
  \newcommand{\xxxParagraphStar}[1]{\oldparagraph*{#1}\mbox{}}
  \newcommand{\xxxParagraphNoStar}[1]{\oldparagraph{#1}\mbox{}}
\fi
\ifx\subparagraph\undefined\else
  \let\oldsubparagraph\subparagraph
  \renewcommand{\subparagraph}{
    \@ifstar
      \xxxSubParagraphStar
      \xxxSubParagraphNoStar
  }
  \newcommand{\xxxSubParagraphStar}[1]{\oldsubparagraph*{#1}\mbox{}}
  \newcommand{\xxxSubParagraphNoStar}[1]{\oldsubparagraph{#1}\mbox{}}
\fi
\makeatother

\usepackage{color}
\usepackage{fancyvrb}
\newcommand{\VerbBar}{|}
\newcommand{\VERB}{\Verb[commandchars=\\\{\}]}
\DefineVerbatimEnvironment{Highlighting}{Verbatim}{commandchars=\\\{\}}
% Add ',fontsize=\small' for more characters per line
\usepackage{framed}
\definecolor{shadecolor}{RGB}{241,243,245}
\newenvironment{Shaded}{\begin{snugshade}}{\end{snugshade}}
\newcommand{\AlertTok}[1]{\textcolor[rgb]{0.68,0.00,0.00}{#1}}
\newcommand{\AnnotationTok}[1]{\textcolor[rgb]{0.37,0.37,0.37}{#1}}
\newcommand{\AttributeTok}[1]{\textcolor[rgb]{0.40,0.45,0.13}{#1}}
\newcommand{\BaseNTok}[1]{\textcolor[rgb]{0.68,0.00,0.00}{#1}}
\newcommand{\BuiltInTok}[1]{\textcolor[rgb]{0.00,0.23,0.31}{#1}}
\newcommand{\CharTok}[1]{\textcolor[rgb]{0.13,0.47,0.30}{#1}}
\newcommand{\CommentTok}[1]{\textcolor[rgb]{0.37,0.37,0.37}{#1}}
\newcommand{\CommentVarTok}[1]{\textcolor[rgb]{0.37,0.37,0.37}{\textit{#1}}}
\newcommand{\ConstantTok}[1]{\textcolor[rgb]{0.56,0.35,0.01}{#1}}
\newcommand{\ControlFlowTok}[1]{\textcolor[rgb]{0.00,0.23,0.31}{\textbf{#1}}}
\newcommand{\DataTypeTok}[1]{\textcolor[rgb]{0.68,0.00,0.00}{#1}}
\newcommand{\DecValTok}[1]{\textcolor[rgb]{0.68,0.00,0.00}{#1}}
\newcommand{\DocumentationTok}[1]{\textcolor[rgb]{0.37,0.37,0.37}{\textit{#1}}}
\newcommand{\ErrorTok}[1]{\textcolor[rgb]{0.68,0.00,0.00}{#1}}
\newcommand{\ExtensionTok}[1]{\textcolor[rgb]{0.00,0.23,0.31}{#1}}
\newcommand{\FloatTok}[1]{\textcolor[rgb]{0.68,0.00,0.00}{#1}}
\newcommand{\FunctionTok}[1]{\textcolor[rgb]{0.28,0.35,0.67}{#1}}
\newcommand{\ImportTok}[1]{\textcolor[rgb]{0.00,0.46,0.62}{#1}}
\newcommand{\InformationTok}[1]{\textcolor[rgb]{0.37,0.37,0.37}{#1}}
\newcommand{\KeywordTok}[1]{\textcolor[rgb]{0.00,0.23,0.31}{\textbf{#1}}}
\newcommand{\NormalTok}[1]{\textcolor[rgb]{0.00,0.23,0.31}{#1}}
\newcommand{\OperatorTok}[1]{\textcolor[rgb]{0.37,0.37,0.37}{#1}}
\newcommand{\OtherTok}[1]{\textcolor[rgb]{0.00,0.23,0.31}{#1}}
\newcommand{\PreprocessorTok}[1]{\textcolor[rgb]{0.68,0.00,0.00}{#1}}
\newcommand{\RegionMarkerTok}[1]{\textcolor[rgb]{0.00,0.23,0.31}{#1}}
\newcommand{\SpecialCharTok}[1]{\textcolor[rgb]{0.37,0.37,0.37}{#1}}
\newcommand{\SpecialStringTok}[1]{\textcolor[rgb]{0.13,0.47,0.30}{#1}}
\newcommand{\StringTok}[1]{\textcolor[rgb]{0.13,0.47,0.30}{#1}}
\newcommand{\VariableTok}[1]{\textcolor[rgb]{0.07,0.07,0.07}{#1}}
\newcommand{\VerbatimStringTok}[1]{\textcolor[rgb]{0.13,0.47,0.30}{#1}}
\newcommand{\WarningTok}[1]{\textcolor[rgb]{0.37,0.37,0.37}{\textit{#1}}}

\usepackage{longtable,booktabs,array}
\usepackage{calc} % for calculating minipage widths
% Correct order of tables after \paragraph or \subparagraph
\usepackage{etoolbox}
\makeatletter
\patchcmd\longtable{\par}{\if@noskipsec\mbox{}\fi\par}{}{}
\makeatother
% Allow footnotes in longtable head/foot
\IfFileExists{footnotehyper.sty}{\usepackage{footnotehyper}}{\usepackage{footnote}}
\makesavenoteenv{longtable}
\usepackage{graphicx}
\makeatletter
\newsavebox\pandoc@box
\newcommand*\pandocbounded[1]{% scales image to fit in text height/width
  \sbox\pandoc@box{#1}%
  \Gscale@div\@tempa{\textheight}{\dimexpr\ht\pandoc@box+\dp\pandoc@box\relax}%
  \Gscale@div\@tempb{\linewidth}{\wd\pandoc@box}%
  \ifdim\@tempb\p@<\@tempa\p@\let\@tempa\@tempb\fi% select the smaller of both
  \ifdim\@tempa\p@<\p@\scalebox{\@tempa}{\usebox\pandoc@box}%
  \else\usebox{\pandoc@box}%
  \fi%
}
% Set default figure placement to htbp
\def\fps@figure{htbp}
\makeatother





\setlength{\emergencystretch}{3em} % prevent overfull lines

\providecommand{\tightlist}{%
  \setlength{\itemsep}{0pt}\setlength{\parskip}{0pt}}



 


\KOMAoption{captions}{tableheading}
\makeatletter
\@ifpackageloaded{caption}{}{\usepackage{caption}}
\AtBeginDocument{%
\ifdefined\contentsname
  \renewcommand*\contentsname{Table of contents}
\else
  \newcommand\contentsname{Table of contents}
\fi
\ifdefined\listfigurename
  \renewcommand*\listfigurename{List of Figures}
\else
  \newcommand\listfigurename{List of Figures}
\fi
\ifdefined\listtablename
  \renewcommand*\listtablename{List of Tables}
\else
  \newcommand\listtablename{List of Tables}
\fi
\ifdefined\figurename
  \renewcommand*\figurename{Figure}
\else
  \newcommand\figurename{Figure}
\fi
\ifdefined\tablename
  \renewcommand*\tablename{Table}
\else
  \newcommand\tablename{Table}
\fi
}
\@ifpackageloaded{float}{}{\usepackage{float}}
\floatstyle{ruled}
\@ifundefined{c@chapter}{\newfloat{codelisting}{h}{lop}}{\newfloat{codelisting}{h}{lop}[chapter]}
\floatname{codelisting}{Listing}
\newcommand*\listoflistings{\listof{codelisting}{List of Listings}}
\makeatother
\makeatletter
\makeatother
\makeatletter
\@ifpackageloaded{caption}{}{\usepackage{caption}}
\@ifpackageloaded{subcaption}{}{\usepackage{subcaption}}
\makeatother
\usepackage{bookmark}
\IfFileExists{xurl.sty}{\usepackage{xurl}}{} % add URL line breaks if available
\urlstyle{same}
\hypersetup{
  pdftitle={Introduction to Bioinformatics},
  pdfauthor={William T. Mills IV},
  colorlinks=true,
  linkcolor={blue},
  filecolor={Maroon},
  citecolor={Blue},
  urlcolor={Blue},
  pdfcreator={LaTeX via pandoc}}


\title{Introduction to Bioinformatics}
\author{William T. Mills IV}
\date{2025-04-03}
\begin{document}
\maketitle

\renewcommand*\contentsname{Table of contents}
{
\hypersetup{linkcolor=}
\setcounter{tocdepth}{2}
\tableofcontents
}

\part{Introduction}

This book is a \textbf{\emph{work in progress}}. Please excuse any
disorganization or incomplete pages.

Bioinformatics is the use of computational tools to process biological
data. As science has progressed, the datasets produced during scientific
experiments have become larger and more complicated and therefore the
ability to glean insights from the data has become less and less
straightforward. The growing complexity and size of scientific data has
necessitated an increased reliance on computational tools for storing,
analyzing, and visualizing data. The past 20 years of scientific
research since the complete sequencing of the human genome has seen an
enormous increase in the types of data being generated and the number of
tools used to process them. While the number of available tools is too
numerous to discuss in a single textbook, and there exists great
variability in user preferences, this book will explore some of the most
popular and widely used tools for processing the most common types of
biological data.

\hfill
\includegraphics[width=3.125in,height=\textheight,keepaspectratio]{images/BookCover.png}

This book will explore what types of biological data are out there, what
tools are available to process them, and how you can begin processing
data on your own computer. Even though the main emphasis of this book
will be helping readers understand how to process sequencing data, many
of the ideas and skills learned will be useful for dealing with other
forms of biological data.

\section*{Prerequisites}\label{prerequisites}
\addcontentsline{toc}{section}{Prerequisites}

\markright{Prerequisites}

While this book is intended for readers with no computational
experience, a general understanding of molecular biology (such as would
be learned in an introductory biology course) would be useful for
understanding the types of data being processed and the analyses being
performed. When possible, references will be included to direct readers
to additional resources for understanding biological concepts.

\section*{Structure of the Book}\label{structure-of-the-book}
\addcontentsline{toc}{section}{Structure of the Book}

\markright{Structure of the Book}

This book is broken up into \_\_\_ main sections:

\begin{enumerate}
\def\labelenumi{\arabic{enumi}.}
\item
  Section 1
\item
  Section 2
\item
  Section 3
\item
  Section 4
\item
  Section 5
\end{enumerate}

Each section contains several chapters that dive into the concepts that
comprise that section. Each chapter will include references to
additional resources that readers may pursue for more advanced
information about the topic being discussed. Additionally, exercises
will be included in each chapter to help readers put newly learned
skills to the test and see real-world applications for the topics being
learned.

\section*{License}\label{license}
\addcontentsline{toc}{section}{License}

\markright{License}

This work is free to use, and is licensed under a Creative Commons
Attribution-NonCommercial-NoDerivatives 4.0 International License.

\section*{About Me}\label{about-me}
\addcontentsline{toc}{section}{About Me}

\markright{About Me}

I earned my B.S. in Biochemistry from the University of Virginia
(Charlottesville, VA) in 2017 and my Ph.D.~in Biological Chemistry from
the Johns Hopkins University School of Medicine (Baltimore, MD) in 2023.
I am now an Assistant Professor of Biology at Mount St.~Mary's
University (Emmitsburg, MD). I began working in bioinformatics during
graduate school where I developed pipelines for processing novel types
of sequencing data. While my formal training was in chemistry and
biology, my experience in bioinformatics was largely self taught. I now
strive to share what I've learned with scientists looking to step into
bioinformatics for the first time.

\hfill
\includegraphics[width=2.08333in,height=\textheight,keepaspectratio]{images/Headshot.jpg}

Please contribute to this book by submitting your feedback:
\url{https://github.com/williamtmills/Introduction-to-Bioinformatics}

This is a Quarto book. To learn more about Quarto books visit
\url{https://quarto.org/docs/books}.

\part{Command Line}

\chapter*{Getting Started}\label{getting-started}
\addcontentsline{toc}{chapter}{Getting Started}

\markboth{Getting Started}{Getting Started}

In order to perform most bioinformatic analyses, users will need access
to a Unix or Unix-like operating system. Windows users will need to
install Windows Subsystem for Linux (WSL) while Mac users just need
access to the Terminal application. Below are instructions for accessing
a Unix or Unix-like command line for Windows and Mac users.

\section*{Windows (using Windows Subsystem for Linux
(WSL))}\label{windows-using-windows-subsystem-for-linux-wsl}
\addcontentsline{toc}{section}{Windows (using Windows Subsystem for
Linux (WSL))}

\markright{Windows (using Windows Subsystem for Linux (WSL))}

\begin{enumerate}
\def\labelenumi{\arabic{enumi}.}
\tightlist
\item
  Open the Windows Terminal application
\item
  Run (i.e.~type in) \texttt{wsl\ -\/-install} (and click enter)
\item
  Restart your device
\item
  Open Windows Terminal
\item
  Run \texttt{wsl}
\item
  By default this should open the distribution (Ubuntu by default)
\end{enumerate}

\begin{itemize}
\item
  If this doesn't work, you may need to run
  \texttt{wsl.exe\ -\/-install\ -d\ Ubuntu} and may need to restart your
  computer again
\item
  Enter a username and password (does not need to match your computer
  login)

  \begin{itemize}
  \tightlist
  \item
    You won't see what you type in appear on the screen, just hit enter
    when you are done
  \end{itemize}
\item
  To log out of the distribution, run \texttt{exit}
\item
  Every time you open Windows Terminal, you need to run \texttt{wsl} to
  turn on Windows Subsystem for Linux
\end{itemize}

\section*{Mac (using Terminal)}\label{mac-using-terminal}
\addcontentsline{toc}{section}{Mac (using Terminal)}

\markright{Mac (using Terminal)}

\begin{enumerate}
\def\labelenumi{\arabic{enumi}.}
\tightlist
\item
  Open Terminal (Applications \textgreater{} Utilities \textgreater{}
  Terminal)
\end{enumerate}

\chapter*{Command Line Tips}\label{command-line-tips}
\addcontentsline{toc}{chapter}{Command Line Tips}

\markboth{Command Line Tips}{Command Line Tips}

\begin{itemize}
\item
  Press `tab' button to autofill
\item
  Depending on shell, hitting tab once (zsh) or twice (bash) will
  display options if multiple exist
\item
  Press `up' arrow to recall last command
\item
  `Control + A' to move to beginning of line
\item
  `Control + E' to move to end of line
\item
  `Control + C' to end (i.e.~kill) command
\item
  `Control + Z' to exit manual
\item
  Avoid using spaces when naming files or folders
\item
  Beware curly (i.e.~smart) quotes \texttt{‘}, be sure to use straight
  quotes \texttt{\textquotesingle{}}. Curly quotes usually come from
  text editors like Microsoft Word and don't work when coding. To ensure
  you are using straight quotes, write code in a plain-text editor like
  TextEdit (Mac) or Notepad (Windows) and turn off spelling and grammar
  checking.
\end{itemize}

\chapter*{Scripting Tips}\label{scripting-tips}
\addcontentsline{toc}{chapter}{Scripting Tips}

\markboth{Scripting Tips}{Scripting Tips}

\begin{itemize}
\item
  When using a bash shell, start every script with \texttt{\#!/bin/bash}
\item
  Write scripts with a plain-text editor such as TextEdit (Mac) or
  Notepad (Windows)
\item
  Turn off spelling and grammar checking:

  \begin{itemize}
  \item
    In TextEdit (Mac): Edit \textgreater{} Spelling and Grammar
    \textgreater{} Uncheck `Check Spelling While Typing' and `Correct
    Spelling Automatically'
  \item
    In Notepad (Windows): Select Setting (top right) \textgreater{}
    Scroll down to Spelling section \textgreater{} Toggle off
    `Autocorrect' and `Spell Check'
  \end{itemize}
\item
  Thoroughly annotate scripts (adding a \texttt{\#} at the beginning of
  a line causes it to be ignored when the computer reads the script).
\end{itemize}

\chapter*{Built-In Command Line
Tools}\label{built-in-command-line-tools}
\addcontentsline{toc}{chapter}{Built-In Command Line Tools}

\markboth{Built-In Command Line Tools}{Built-In Command Line Tools}

Your terminal will have several commands already built into the command
line. Below are a few of them with examples of how they are used.

\texttt{pwd}: print working directory (where am I currently?)

~

\texttt{/}: root directory (folder that holds all other folders)\\
\texttt{./}: current directory (current folder)\\
\texttt{../}: parent directory (folder containing your current folder)

~

\texttt{ls}: list directory contents (list everything in current folder)
(\href{https://www.geeksforgeeks.org/ls-command-in-linux/}{usage})

\begin{itemize}
\item
  Flags:

  \begin{itemize}
  \item
    \texttt{-l}: long format (permissions, sizes, dates, etc.)
  \item
    \texttt{-F}: indicate item types

    \begin{itemize}
    \item
      trailing / = directory
    \item
      @ = link
    \item
      * = executable
    \end{itemize}
  \item
    \texttt{-r}: reverse order (reverse alphabetic order)
  \end{itemize}
\end{itemize}

~

\texttt{mkdir}: make directory (i.e.~folder)
(\href{https://www.geeksforgeeks.org/mkdir-command-in-linux-with-examples/}{usage})

~

\texttt{mv}: move object
(\href{https://www.geeksforgeeks.org/mv-command-linux-examples/}{usage})

\begin{itemize}
\item
  \texttt{mv\ {[}source(s){]}\ {[}destination{]}}
\item
  \textbf{*Warning*} Will overwrite object if name already exists
\item
  When in the same folder: renames object
\item
  Flags:

  \begin{itemize}
  \tightlist
  \item
    \texttt{-i}: interactive (ask for confirmation before overwriting)
  \end{itemize}
\end{itemize}

~

\texttt{cp}: copy object
(\href{https://www.geeksforgeeks.org/cp-command-linux-examples/}{usage})

\begin{itemize}
\item
  cp {[}source object(s){]} {[}destination{]}
\item
  \textbf{*Warning*} Will overwrite object if name already exists
\item
  Flags:

  \begin{itemize}
  \item
    -R: copy directory (i.e.~folder)
  \item
    -i: interactive (ask for confirmation before overwriting)
  \end{itemize}
\end{itemize}

~

\texttt{head}: display the beginning (i.e.~top) of a file
(\href{https://www.geeksforgeeks.org/head-command-linux-examples/}{usage})

\begin{itemize}
\item
  Flags:

  \begin{itemize}
  \item
    \texttt{-1}: display first row
  \item
    \texttt{-3}: display first 3 rows
  \end{itemize}
\end{itemize}

~

\texttt{tail}: display the end (i.e.~bottom) of a file
(\href{https://www.geeksforgeeks.org/tail-command-linux-examples/}{usage})

\begin{itemize}
\item
  Flags:

  \begin{itemize}
  \item
    \texttt{-1}: display last row
  \item
    \texttt{-3}: display last 3 rows
  \end{itemize}
\end{itemize}

~

\texttt{\textbar{}} (shift + backslash): pipe (use outputs from one
command as inputs for another)

\begin{itemize}
\item
  Example: display the 10th row of a file

  \begin{itemize}
  \tightlist
  \item
    \texttt{head\ -10\ path/to/file\ \textbar{}\ tail\ -1}
  \end{itemize}
\end{itemize}

~

\texttt{wc}: word count (lines, words, characters)
(\href{https://www.geeksforgeeks.org/wc-command-linux-examples/}{usage})

\begin{itemize}
\item
  Flags:

  \begin{itemize}
  \item
    \texttt{-l}: lines
  \item
    \texttt{-w}: words
  \item
    \texttt{-c}: characters
  \end{itemize}
\end{itemize}

~

\chapter*{Python Example 2}\label{python-example-2}
\addcontentsline{toc}{chapter}{Python Example 2}

\markboth{Python Example 2}{Python Example 2}

\section*{Try This Interactive Python
Code}\label{try-this-interactive-python-code}
\addcontentsline{toc}{section}{Try This Interactive Python Code}

\markright{Try This Interactive Python Code}

Click ``Run Code'' to execute:

\begin{Shaded}
\begin{Highlighting}[]
\NormalTok{for x in range(12):}
\NormalTok{  print(x ** 2)}
\end{Highlighting}
\end{Shaded}

\chapter*{awk}\label{awk}
\addcontentsline{toc}{chapter}{awk}

\markboth{awk}{awk}

awk is scripting language named after its developers (Aho, Weinberger,
and Kernighan)
(\href{https://www.geeksforgeeks.org/awk-command-unixlinux-examples/}{usage})

\begin{center}\rule{0.5\linewidth}{0.5pt}\end{center}

Things to know about awk:

\begin{itemize}
\item
  awk refers to columns as field, such as in the variables for number of
  fields (\texttt{NF}), input field separator (\texttt{FS}), and output
  field separator (\texttt{OFS}).
\item
  awk refers to rows as records, such as in the variables for record
  number (\texttt{NR}), input record separator (\texttt{RS}), and output
  record separator (\texttt{ORS}).
\item
  By default, awk recognizes a space or tab as a field separator. If
  your input file has field separators other than a space or a tab, you
  need to specify it using the \texttt{-F} flag.
\item
  awk has several built-in variables that can be used when writing code:

  \begin{itemize}
  \item
    \texttt{\$1} = field 1 (\$2 = field 2, \$3 = field 3, \ldots)
  \item
    \texttt{\$0} = entire record
  \item
    \texttt{NF} = number of fields
  \item
    \texttt{NR} = number of records
  \item
    \texttt{FS} = input field separator; default is white space
    (i.e.~space and tab)
  \item
    \texttt{OFS} = output field separator; default is single space
  \item
    \texttt{RS} = input record separator; default is new line
  \item
    \texttt{ORS} = output record separator; default is new line
  \item
    \texttt{{[}0-9{]}} = any number
  \end{itemize}
\end{itemize}

\begin{center}\rule{0.5\linewidth}{0.5pt}\end{center}

Below are examples of how the awk command can be used to achieve lots of
desired outcomes when processing data files.

\begin{Shaded}
\begin{Highlighting}[]
\FunctionTok{cat}\NormalTok{ data.txt }\CommentTok{\# View data.txt}
\end{Highlighting}
\end{Shaded}

\begin{verbatim}
Genes   Sample1 Sample2 Sample3
GeneX   3210    5678    689
GeneY   2354    6700    987
GeneZ   2315    7890    123
\end{verbatim}

Printing all of the fields (synonymous with
\texttt{awk\ \textquotesingle{}\{print\ \$0\}\textquotesingle{}\ data.txt})

\begin{Shaded}
\begin{Highlighting}[]
\FunctionTok{awk} \StringTok{\textquotesingle{}\{print\}\textquotesingle{}}\NormalTok{ data.txt}
\end{Highlighting}
\end{Shaded}

\begin{verbatim}
Genes   Sample1 Sample2 Sample3
GeneX   3210    5678    689
GeneY   2354    6700    987
GeneZ   2315    7890    123
\end{verbatim}

Print a particular field (e.g.~field 1)

\begin{Shaded}
\begin{Highlighting}[]
\FunctionTok{awk} \StringTok{\textquotesingle{}\{print $1\}\textquotesingle{}}\NormalTok{ data.txt}
\end{Highlighting}
\end{Shaded}

\begin{verbatim}
Genes
GeneX
GeneY
GeneZ
\end{verbatim}

If input file uses a comma (\texttt{,}) as a field separator instead of
space or tab, set input field separator as (\texttt{,}) (synonymous with
\texttt{awk\ \textquotesingle{}\{\ FS\ =\ ","\ \}\ ;\ \{print\ \$1\}\textquotesingle{}\ data.csv}).

\begin{quote}
Try leaving out the \texttt{-F,} and see what happens.
\end{quote}

\begin{Shaded}
\begin{Highlighting}[]
\FunctionTok{cat}\NormalTok{ data.csv }\CommentTok{\# View comma separated file (.csv)}
\end{Highlighting}
\end{Shaded}

\begin{verbatim}
Genes,Sample1,Sample2,Sample3
GeneX,3210,5678,689
GeneY,2354,6700,987
GeneZ,2315,7890,123
\end{verbatim}

\begin{Shaded}
\begin{Highlighting}[]
\FunctionTok{awk} \AttributeTok{{-}F,} \StringTok{\textquotesingle{}\{print $1\}\textquotesingle{}}\NormalTok{ data.csv}
\end{Highlighting}
\end{Shaded}

\begin{verbatim}
Genes
GeneX
GeneY
GeneZ
\end{verbatim}

Print multiple fields (e.g.~field 1 and 3)

\begin{Shaded}
\begin{Highlighting}[]
\FunctionTok{awk} \StringTok{\textquotesingle{}\{print $1,$3\}\textquotesingle{}}\NormalTok{ data.txt}
\end{Highlighting}
\end{Shaded}

\begin{verbatim}
Genes Sample2
GeneX 5678
GeneY 6700
GeneZ 7890
\end{verbatim}

Print the last field

\begin{Shaded}
\begin{Highlighting}[]
\FunctionTok{awk} \StringTok{\textquotesingle{}\{print $NF\}\textquotesingle{}}\NormalTok{ data.txt}
\end{Highlighting}
\end{Shaded}

\begin{verbatim}
Sample3
689
987
123
\end{verbatim}

Print all records after the first record (synonymous with
\texttt{awk\ \textquotesingle{}NR!=1\ \{print\}\textquotesingle{}\ /path/to/file})

\begin{Shaded}
\begin{Highlighting}[]
\FunctionTok{awk} \StringTok{\textquotesingle{}NR\textgreater{}1 \{print\}\textquotesingle{}}\NormalTok{ data.txt}
\end{Highlighting}
\end{Shaded}

\begin{verbatim}
GeneX   3210    5678    689
GeneY   2354    6700    987
GeneZ   2315    7890    123
\end{verbatim}

Print a particular record (e.g.~record 3)

\begin{Shaded}
\begin{Highlighting}[]
\FunctionTok{awk} \StringTok{\textquotesingle{}NR==3 \{print\}\textquotesingle{}}\NormalTok{ data.txt}
\end{Highlighting}
\end{Shaded}

\begin{verbatim}
GeneY   2354    6700    987
\end{verbatim}

Print all records except for a particular record (e.g.~not record 3)

\begin{Shaded}
\begin{Highlighting}[]
\FunctionTok{awk} \StringTok{\textquotesingle{}NR!=3 \{print\}\textquotesingle{}}\NormalTok{ data.txt}
\end{Highlighting}
\end{Shaded}

\begin{verbatim}
Genes   Sample1 Sample2 Sample3
GeneX   3210    5678    689
GeneZ   2315    7890    123
\end{verbatim}

Print a range of records (e.g.~records 2 to 3)

\begin{Shaded}
\begin{Highlighting}[]
\FunctionTok{awk} \StringTok{\textquotesingle{}NR==2, NR==3 \{print\}\textquotesingle{}}\NormalTok{ data.txt}
\end{Highlighting}
\end{Shaded}

\begin{verbatim}
GeneX   3210    5678    689
GeneY   2354    6700    987
\end{verbatim}

Print records with fewer than a certain number of fields (e.g.~fewer
than 4 fields)

\begin{Shaded}
\begin{Highlighting}[]
\FunctionTok{cat}\NormalTok{ data2.txt }\CommentTok{\# View data2.txt}
\end{Highlighting}
\end{Shaded}

\begin{verbatim}
Genes   Sample1 Sample2 Sample3
GeneX   3210    5678    689
GeneY   2354    6700    
GeneZ   2315    7890    123
\end{verbatim}

\begin{Shaded}
\begin{Highlighting}[]
\FunctionTok{awk} \StringTok{\textquotesingle{}NF\textless{}4 \{print\}\textquotesingle{}}\NormalTok{ data2.txt}
\end{Highlighting}
\end{Shaded}

\begin{verbatim}
GeneY   2354    6700    
\end{verbatim}

Print records containing a certain string anywhere in record
(e.g.~\texttt{abc})

\begin{Shaded}
\begin{Highlighting}[]
\FunctionTok{cat}\NormalTok{ data3.txt}
\end{Highlighting}
\end{Shaded}

\begin{verbatim}
abcd    dcba    efgh    aabc
bcde    dabc    cbad    abdc
cdef    defg    efgh    fghi
\end{verbatim}

\begin{Shaded}
\begin{Highlighting}[]
\FunctionTok{awk} \StringTok{\textquotesingle{}/abc/ \{print\}\textquotesingle{}}\NormalTok{ data3.txt}
\end{Highlighting}
\end{Shaded}

\begin{verbatim}
abcd    dcba    efgh    aabc
bcde    dabc    cbad    abdc
\end{verbatim}

Print records starting with a certain string (e.g.~\texttt{abc})

\begin{Shaded}
\begin{Highlighting}[]
\FunctionTok{awk} \StringTok{\textquotesingle{}/\^{}abc/ \{print\}\textquotesingle{}}\NormalTok{ data3.txt}
\end{Highlighting}
\end{Shaded}

\begin{verbatim}
abcd    dcba    efgh    aabc
\end{verbatim}

Print records ending with a certain string (e.g.~\texttt{abc})
\textgreater One caveat between macOS and Windows (even when using wsl)
is that the line ending character in macOS (i.e.~unix) is
\texttt{\textbackslash{}n} while the line ending character in Windows is
\texttt{\textbackslash{}r\textbackslash{}n}. This means that a text file
made on a Mac may have a different line ending character than Windows
recognizes (and vice versa). To avoid this problem\ldots{}

\begin{Shaded}
\begin{Highlighting}[]
\FunctionTok{awk} \StringTok{\textquotesingle{}/abc$/ \{print\}\textquotesingle{}}\NormalTok{ data3.txt}
\end{Highlighting}
\end{Shaded}

\begin{verbatim}
abcd    dcba    efgh    aabc
\end{verbatim}

Print records that don't contain a certain string anywhere in record
(e.g.~\texttt{abc})

\begin{Shaded}
\begin{Highlighting}[]
\FunctionTok{awk} \StringTok{\textquotesingle{}!/abc/ \{print\}\textquotesingle{}}\NormalTok{ data3.txt}
\end{Highlighting}
\end{Shaded}

\begin{verbatim}
cdef    defg    efgh    fghi
\end{verbatim}

Print records that don't start with a certain string (e.g.~\texttt{abc})

\begin{Shaded}
\begin{Highlighting}[]
\FunctionTok{awk} \StringTok{\textquotesingle{}!/\^{}abc/ \{print\}\textquotesingle{}}\NormalTok{ data3.txt}
\end{Highlighting}
\end{Shaded}

\begin{verbatim}
bcde    dabc    cbad    abdc
cdef    defg    efgh    fghi
\end{verbatim}

Print records that don't end with a certain string (e.g.~\texttt{abc})

\begin{Shaded}
\begin{Highlighting}[]
\FunctionTok{awk} \StringTok{\textquotesingle{}!/abc$/ \{print\}\textquotesingle{}}\NormalTok{ data3.txt}
\end{Highlighting}
\end{Shaded}

\begin{verbatim}
bcde    dabc    cbad    abdc
cdef    defg    efgh    fghi
\end{verbatim}

Print records where a particular field contains a string
(e.g.~\texttt{abc} in field 1)

\begin{Shaded}
\begin{Highlighting}[]
\FunctionTok{awk} \StringTok{\textquotesingle{}$1 \textasciitilde{} /abc/ \{print\}\textquotesingle{}}\NormalTok{ data3.txt}
\end{Highlighting}
\end{Shaded}

\begin{verbatim}
abcd    dcba    efgh    aabc
\end{verbatim}

Print records where a particular field starts with a string
(e.g.~\texttt{abc} in field 1)

\begin{Shaded}
\begin{Highlighting}[]
\FunctionTok{awk} \StringTok{\textquotesingle{}$1 \textasciitilde{} /\^{}abc/ \{print\}\textquotesingle{}}\NormalTok{ data3.txt}
\end{Highlighting}
\end{Shaded}

\begin{verbatim}
abcd    dcba    efgh    aabc
\end{verbatim}

Print records where a particular field ends with a string
(e.g.~\texttt{abc} in field 4)

\begin{Shaded}
\begin{Highlighting}[]
\FunctionTok{awk} \StringTok{\textquotesingle{}$4 \textasciitilde{} /abc$/ \{print\}\textquotesingle{}}\NormalTok{ data3.txt}
\end{Highlighting}
\end{Shaded}

\begin{verbatim}
abcd    dcba    efgh    aabc
\end{verbatim}

Print records where a particular field starts with any number
(e.g.~field 1)

\begin{Shaded}
\begin{Highlighting}[]
\FunctionTok{cat}\NormalTok{ data4.txt}
\end{Highlighting}
\end{Shaded}

\begin{verbatim}
1ABC    D1CB    EF1G    AAB1
b2cd    da2b    cba2    2abc
CD3E    DEF3    3EFG    F2GH
\end{verbatim}

\begin{Shaded}
\begin{Highlighting}[]
\FunctionTok{awk} \StringTok{\textquotesingle{}$1 \textasciitilde{} /\^{}[0{-}9]/ \{print\}\textquotesingle{}}\NormalTok{ data4.txt}
\end{Highlighting}
\end{Shaded}

\begin{verbatim}
1ABC    D1CB    EF1G    AAB1
\end{verbatim}

Print records where a particular field ends with any number (e.g.~field
1)

\begin{Shaded}
\begin{Highlighting}[]
\FunctionTok{awk} \StringTok{\textquotesingle{}$1 \textasciitilde{} /[0{-}9]$/ \{print\}\textquotesingle{}}\NormalTok{ data4.txt}
\end{Highlighting}
\end{Shaded}

Ignore case when looking for records containing a string
(e.g.~\texttt{abc})

\begin{Shaded}
\begin{Highlighting}[]
\FunctionTok{awk} \StringTok{\textquotesingle{}tolower($0) \textasciitilde{} /abc/ \{print\}\textquotesingle{}}\NormalTok{ data4.txt}
\end{Highlighting}
\end{Shaded}

\begin{verbatim}
1ABC    D1CB    EF1G    AAB1
b2cd    da2b    cba2    2abc
\end{verbatim}

Print records that contain a certain value in a particular field
(e.g.~the number 3210 in field 2)

\begin{Shaded}
\begin{Highlighting}[]
\FunctionTok{cat}\NormalTok{ data.txt}
\end{Highlighting}
\end{Shaded}

\begin{verbatim}
Genes   Sample1 Sample2 Sample3
GeneX   3210    5678    689
GeneY   2354    6700    987
GeneZ   2315    7890    123
\end{verbatim}

\begin{Shaded}
\begin{Highlighting}[]
\FunctionTok{awk} \StringTok{\textquotesingle{}$2==3210 \{print\}\textquotesingle{}}\NormalTok{ data.txt}
\end{Highlighting}
\end{Shaded}

\begin{verbatim}
GeneX   3210    5678    689
\end{verbatim}

Print records that do not contain a certain value in a particular field
(e.g.~not the number 10 in field 2)

\begin{Shaded}
\begin{Highlighting}[]
\FunctionTok{awk} \StringTok{\textquotesingle{}$2!=3210 \{print\}\textquotesingle{}}\NormalTok{ data.txt}
\end{Highlighting}
\end{Shaded}

\begin{verbatim}
Genes   Sample1 Sample2 Sample3
GeneY   2354    6700    987
GeneZ   2315    7890    123
\end{verbatim}

Print records that contain a value greater than a certain value in a
particular field (e.g.~\textgreater2354 in field 2)

\begin{Shaded}
\begin{Highlighting}[]
\FunctionTok{awk} \StringTok{\textquotesingle{}$2\textgreater{}2354 \{print\}\textquotesingle{}}\NormalTok{ data.txt}
\end{Highlighting}
\end{Shaded}

\begin{verbatim}
Genes   Sample1 Sample2 Sample3
GeneX   3210    5678    689
\end{verbatim}

Print records that contain a value less than a certain value in a
particular field (e.g.~\textless2354 in field 2)

\begin{Shaded}
\begin{Highlighting}[]
\FunctionTok{awk} \StringTok{\textquotesingle{}$2\textless{}2354 \{print\}\textquotesingle{}}\NormalTok{ data.txt}
\end{Highlighting}
\end{Shaded}

\begin{verbatim}
GeneZ   2315    7890    123
\end{verbatim}

Print records that contain a value less than or equal to a certain value
in a particular field (e.g.~\textless2354 in field 2)

\begin{Shaded}
\begin{Highlighting}[]
\FunctionTok{awk} \StringTok{\textquotesingle{}$2\textless{}=2354 \{print\}\textquotesingle{}}\NormalTok{ data.txt}
\end{Highlighting}
\end{Shaded}

\begin{verbatim}
GeneY   2354    6700    987
GeneZ   2315    7890    123
\end{verbatim}

Sum values in a field (e.g.~field 2)

\begin{Shaded}
\begin{Highlighting}[]
\FunctionTok{awk} \StringTok{\textquotesingle{}\{sum+=$2;\} END\{print sum;\}\textquotesingle{}}\NormalTok{ data.txt}
\end{Highlighting}
\end{Shaded}

\begin{verbatim}
7879
\end{verbatim}

Remember to add \texttt{NR\textgreater{}1} if your file has a header in
case the headers are numeric

\begin{Shaded}
\begin{Highlighting}[]
\FunctionTok{awk} \StringTok{\textquotesingle{}NR\textgreater{}1 \{sum+=$2;\} END\{print sum;\}\textquotesingle{}}\NormalTok{ data.txt}
\end{Highlighting}
\end{Shaded}

\begin{verbatim}
7879
\end{verbatim}

Remove blank lines

\begin{Shaded}
\begin{Highlighting}[]
\FunctionTok{cat}\NormalTok{ data5.txt}
\end{Highlighting}
\end{Shaded}

\begin{verbatim}
Genes   Sample1 Sample2 Sample3
GeneX   3210    5678    689
GeneY   2354    6700    987

GeneZ   2315    7890    123
\end{verbatim}

\begin{Shaded}
\begin{Highlighting}[]
\FunctionTok{awk} \StringTok{\textquotesingle{}NF\textquotesingle{}}\NormalTok{ data5.txt}
\end{Highlighting}
\end{Shaded}

\begin{verbatim}
Genes   Sample1 Sample2 Sample3
GeneX   3210    5678    689
GeneY   2354    6700    987
GeneZ   2315    7890    123
\end{verbatim}

Print the record number at beginning of record

\begin{Shaded}
\begin{Highlighting}[]
\FunctionTok{awk} \StringTok{\textquotesingle{}\{print NR,$0\}\textquotesingle{}}\NormalTok{ data.txt}
\end{Highlighting}
\end{Shaded}

\begin{verbatim}
1 Genes Sample1 Sample2 Sample3
2 GeneX 3210    5678    689
3 GeneY 2354    6700    987
4 GeneZ 2315    7890    123
\end{verbatim}

\chapter{Python Example}\label{python-example}

\begin{Shaded}
\begin{Highlighting}[]
\NormalTok{for x in range(6):}
\NormalTok{  print(x ** 2)}
\end{Highlighting}
\end{Shaded}

\part{R}

\chapter*{Installing R and RStudio}\label{installing-r-and-rstudio}
\addcontentsline{toc}{chapter}{Installing R and RStudio}

\markboth{Installing R and RStudio}{Installing R and RStudio}

R is a free software environment for statistical computing and graphics
that is very popular for processing and visualizing biological data.
RStudio integrated development environment (IDE) is a set of tools built
to help you be more productive with R (and Python).

\begin{enumerate}
\def\labelenumi{\arabic{enumi}.}
\item
  Downloaded R: \url{https://cran.rstudio.com/}
\item
  Download RStudio Desktop:
  \url{https://posit.co/download/rstudio-desktop/}
\end{enumerate}

Alternatively, RStudio can be used \href{https://posit.cloud/}{online},
however this version may have difficulty with more computationally-heavy
processes.

\chapter*{Data Types}\label{data-types}
\addcontentsline{toc}{chapter}{Data Types}

\markboth{Data Types}{Data Types}

R stores the data you give it in the most optimal way that it can for it
to do the manipulations and graphing that you desire. Here are the terms
we use to describe these classifications:

\begin{longtable}[]{@{}ll@{}}
\toprule\noalign{}
Term & Definition \\
\midrule\noalign{}
\endhead
\bottomrule\noalign{}
\endlastfoot
Type & Classification for how an object is stored internally in R
(a.k.a. storage mode) \\
Class & Broader classification of how an object is stored in R (the more
common language you would use in conversation and writing) \\
\end{longtable}

\begin{quote}
The term ``object'' describes any ``thing'' in R such as numbers,
characters, etc. (described below).
\end{quote}

You can figure out the type or class of any object by using the
\texttt{typeof()} or \texttt{class()} functions, respectively. Here is
how we determine the type and class of the object \texttt{3}:

\begin{Shaded}
\begin{Highlighting}[]
\FunctionTok{typeof}\NormalTok{(}\DecValTok{3}\NormalTok{)}
\end{Highlighting}
\end{Shaded}

\begin{verbatim}
[1] "double"
\end{verbatim}

\begin{Shaded}
\begin{Highlighting}[]
\FunctionTok{class}\NormalTok{(}\DecValTok{3}\NormalTok{)}
\end{Highlighting}
\end{Shaded}

\begin{verbatim}
[1] "numeric"
\end{verbatim}

\begin{quote}
These operations are called functions. A function is some task you want
R to perform and what you put inside the parentheses is the object you
want R to perform that function on. In these examples, the ``task'' you
want R to perform is to tell you what the type or class of of an object
is and the object in these examples is \texttt{3}.
\end{quote}

\begin{quote}
Every function in R has specific requirements for how it is used. To
learn about the requirements for any particular function you can use the
\texttt{help()} function. For example, to learn how to use the
\texttt{typeof()} function, you can run \texttt{help(typeof)}.
\end{quote}

In the examples above you'll notice that \texttt{typeof()} and
\texttt{class()} use different terms to describe the same object (in
this case, \texttt{3}). Below is a table that helps visualize these
distinctions (there are many other types and classes but these are the
ones you will see most often).

\begin{longtable}[]{@{}
  >{\raggedright\arraybackslash}p{(\linewidth - 6\tabcolsep) * \real{0.2500}}
  >{\raggedright\arraybackslash}p{(\linewidth - 6\tabcolsep) * \real{0.2500}}
  >{\raggedright\arraybackslash}p{(\linewidth - 6\tabcolsep) * \real{0.2500}}
  >{\raggedright\arraybackslash}p{(\linewidth - 6\tabcolsep) * \real{0.2500}}@{}}
\toprule\noalign{}
\begin{minipage}[b]{\linewidth}\raggedright
Object
\end{minipage} & \begin{minipage}[b]{\linewidth}\raggedright
Type
\end{minipage} & \begin{minipage}[b]{\linewidth}\raggedright
Class
\end{minipage} & \begin{minipage}[b]{\linewidth}\raggedright
Notes
\end{minipage} \\
\midrule\noalign{}
\endhead
\bottomrule\noalign{}
\endlastfoot
``a'' & Character & Character & Letters must be in quotes or they will
be interpreted as an object (discussed below) \\
3 & Double & Numeric & Double = double precision floating point numbers
(significand and exponent) \\
3L & Integer & Integer & Can only be whole numbers (i.e.~no decimals) \\
TRUE or FALSE & Logical & Logical & \\
\end{longtable}

In addition to determining the type or class of an object (using
\texttt{typeof()} or \texttt{class()}), you can also determine if an
object is a particular type or class (using \texttt{is.double()},
\texttt{is.numeric()}, \texttt{is.integer()}, \texttt{is.character()},
\texttt{is.logical()}). For example:

\begin{Shaded}
\begin{Highlighting}[]
\FunctionTok{is.double}\NormalTok{(}\DecValTok{3}\NormalTok{) }\CommentTok{\# TRUE}
\FunctionTok{is.numeric}\NormalTok{(}\DecValTok{3}\NormalTok{) }\CommentTok{\# TRUE}
\FunctionTok{is.integer}\NormalTok{(}\DecValTok{3}\NormalTok{) }\CommentTok{\# FALSE}
\FunctionTok{is.character}\NormalTok{(}\DecValTok{3}\NormalTok{) }\CommentTok{\# FALSE}

\FunctionTok{is.double}\NormalTok{(}\DecValTok{3}\NormalTok{L) }\CommentTok{\# FALSE}
\FunctionTok{is.numeric}\NormalTok{(}\DecValTok{3}\NormalTok{L) }\CommentTok{\# TRUE}
\FunctionTok{is.integer}\NormalTok{(}\DecValTok{3}\NormalTok{L) }\CommentTok{\# TRUE}
\FunctionTok{is.character}\NormalTok{(}\DecValTok{3}\NormalTok{L) }\CommentTok{\# FALSE}

\FunctionTok{is.logical}\NormalTok{(}\ConstantTok{TRUE}\NormalTok{) }\CommentTok{\# TRUE}
\FunctionTok{is.numeric}\NormalTok{(}\ConstantTok{TRUE}\NormalTok{) }\CommentTok{\# FALSE}
\FunctionTok{is.character}\NormalTok{(}\ConstantTok{TRUE}\NormalTok{) }\CommentTok{\# FALSE}
\FunctionTok{is.character}\NormalTok{(}\StringTok{"TRUE"}\NormalTok{) }\CommentTok{\# TRUE}

\FunctionTok{is.logical}\NormalTok{(}\StringTok{"abc"}\NormalTok{) }\CommentTok{\# FALSE}
\FunctionTok{is.numeric}\NormalTok{(}\StringTok{"abc"}\NormalTok{) }\CommentTok{\# FALSE}
\FunctionTok{is.character}\NormalTok{(}\StringTok{"abc"}\NormalTok{) }\CommentTok{\# TRUE}
\end{Highlighting}
\end{Shaded}

Notice that the ``answer'' to these functions is logical (\texttt{TRUE}
or \texttt{FALSE}); this is a very useful feature of R that you will see
later.

\begin{quote}
Imagine having a data set with a list of cities and their average yearly
rainfall totals. With a simple line of code, R can quickly tell you
whether each city's average yearly rainfall is above or below a certain
number (outputing \texttt{TRUE} or \texttt{FALSE} for each city).
\end{quote}

It is often useful to store some information (numbers, text, etc.) as a
variable that can be easily referenced later. For example, rather than
having to type out the number 10481034582 every time you need to use it,
you can simply save it as a variable (R calls these variables
``objects''). Let's use the letter \texttt{x} as our new object. After
executing this function you should see it appear in your R environment
under ``Values'' in the Environment panel in the top-right corner of
RStudio:

\begin{Shaded}
\begin{Highlighting}[]
\NormalTok{x }\OtherTok{\textless{}{-}} \DecValTok{10481034582}
\NormalTok{x}
\end{Highlighting}
\end{Shaded}

\begin{verbatim}
[1] 10481034582
\end{verbatim}

You can change the value of your object by running the same code with a
different value:

\begin{Shaded}
\begin{Highlighting}[]
\NormalTok{x }\OtherTok{\textless{}{-}} \DecValTok{54321}
\NormalTok{x}
\end{Highlighting}
\end{Shaded}

\begin{verbatim}
[1] 54321
\end{verbatim}

\begin{quote}
CAUTION: this overwriting of an object can't be undone except by
starting over and regenerating it the way you did originally. To see if
you have used an object name already, check under ``Values'' in the
Environment panel in the top-right of RStudio.
\end{quote}

Now, anytime you need to use that value you can simply use the object
instead:

\begin{Shaded}
\begin{Highlighting}[]
\FunctionTok{typeof}\NormalTok{(}\DecValTok{12345}\NormalTok{)}
\end{Highlighting}
\end{Shaded}

\begin{verbatim}
[1] "double"
\end{verbatim}

\begin{Shaded}
\begin{Highlighting}[]
\NormalTok{x }\OtherTok{\textless{}{-}} \DecValTok{12345}
\FunctionTok{typeof}\NormalTok{(x)}
\end{Highlighting}
\end{Shaded}

\begin{verbatim}
[1] "double"
\end{verbatim}

\begin{Shaded}
\begin{Highlighting}[]
\CommentTok{\# Let\textquotesingle{}s overwrite the object x to prove that it is holding the correct value}

\FunctionTok{typeof}\NormalTok{(}\StringTok{"abcd"}\NormalTok{)}
\end{Highlighting}
\end{Shaded}

\begin{verbatim}
[1] "character"
\end{verbatim}

\begin{Shaded}
\begin{Highlighting}[]
\NormalTok{x }\OtherTok{\textless{}{-}} \StringTok{"abcd"}
\FunctionTok{typeof}\NormalTok{(x)}
\end{Highlighting}
\end{Shaded}

\begin{verbatim}
[1] "character"
\end{verbatim}

These new objects don't have to be single letters, they can also be
multiple letters or words; they just can't start with a number or
contain any special characters other than an underscore \texttt{\_}.
This can help you remember which objects are which:

\begin{Shaded}
\begin{Highlighting}[]
\NormalTok{big\_number }\OtherTok{\textless{}{-}} \DecValTok{10481034582}
\NormalTok{big\_number}
\end{Highlighting}
\end{Shaded}

\begin{verbatim}
[1] 10481034582
\end{verbatim}

\begin{Shaded}
\begin{Highlighting}[]
\NormalTok{small\_number }\OtherTok{\textless{}{-}} \DecValTok{54321}
\NormalTok{small\_number}
\end{Highlighting}
\end{Shaded}

\begin{verbatim}
[1] 54321
\end{verbatim}

As you start generating your own code and pipelines for analyzing data,
you will find this feature particularly useful as you can avoid having
to change every occurrence of an object and instead just change it a
single time when you originally create the object.

\chapter*{Data Structures}\label{data-structures}
\addcontentsline{toc}{chapter}{Data Structures}

\markboth{Data Structures}{Data Structures}

In the previous section we discussed storing objects as variables so
that they can be easily referenced later:

\begin{Shaded}
\begin{Highlighting}[]
\NormalTok{x }\OtherTok{\textless{}{-}} \DecValTok{54321} \CommentTok{\# Store the number 54321 as a object called \textquotesingle{}x\textquotesingle{}}
\NormalTok{x }\CommentTok{\# View the contents of x}
\end{Highlighting}
\end{Shaded}

\begin{verbatim}
[1] 54321
\end{verbatim}

\begin{Shaded}
\begin{Highlighting}[]
\FunctionTok{typeof}\NormalTok{(x) }\CommentTok{\# Check the data type of x}
\end{Highlighting}
\end{Shaded}

\begin{verbatim}
[1] "double"
\end{verbatim}

\begin{Shaded}
\begin{Highlighting}[]
\NormalTok{y }\OtherTok{\textless{}{-}} \StringTok{"science"} \CommentTok{\# Store the word "science" as a object called \textquotesingle{}y\textquotesingle{}}
\NormalTok{y }\CommentTok{\# View the contents of y}
\end{Highlighting}
\end{Shaded}

\begin{verbatim}
[1] "science"
\end{verbatim}

\begin{Shaded}
\begin{Highlighting}[]
\FunctionTok{typeof}\NormalTok{(y) }\CommentTok{\# Check the data type of y}
\end{Highlighting}
\end{Shaded}

\begin{verbatim}
[1] "character"
\end{verbatim}

R can handle much more than a single object in a variable. You can add
thousands of objects to a single variable and they can be of the same or
different types and even have multiple rows or columns. The terms used
to describe how multiple objects are stored depends on whether the
objects are of the same (homogeneous) or different (heterogeneous) types
and their dimensions:

\begin{longtable}[]{@{}lll@{}}
\toprule\noalign{}
Rows & Homogeneous & Heterogeneous \\
\midrule\noalign{}
\endhead
\bottomrule\noalign{}
\endlastfoot
1d & Atomic vector & List \\
2d & Matrix & Data frame \\
nd & Array & \\
\end{longtable}

In the simplest case, a group of objects of the same type can be
combined (using combine, \texttt{c()}) to form an atomic vector:

\begin{Shaded}
\begin{Highlighting}[]
\NormalTok{string }\OtherTok{\textless{}{-}} \FunctionTok{c}\NormalTok{(}\DecValTok{1}\NormalTok{,}\DecValTok{2}\NormalTok{,}\DecValTok{3}\NormalTok{,}\DecValTok{4}\NormalTok{,}\DecValTok{5}\NormalTok{) }\CommentTok{\# Create an object called \textquotesingle{}string\textquotesingle{} containing the numbers 1{-}5}
\NormalTok{string }\CommentTok{\# View the contents of string}
\end{Highlighting}
\end{Shaded}

\begin{verbatim}
[1] 1 2 3 4 5
\end{verbatim}

\begin{Shaded}
\begin{Highlighting}[]
\FunctionTok{typeof}\NormalTok{(string) }\CommentTok{\# Check the data type of string}
\end{Highlighting}
\end{Shaded}

\begin{verbatim}
[1] "double"
\end{verbatim}

\begin{Shaded}
\begin{Highlighting}[]
\FunctionTok{is.atomic}\NormalTok{(string) }\CommentTok{\# Check if string is an atomic vector}
\end{Highlighting}
\end{Shaded}

\begin{verbatim}
[1] TRUE
\end{verbatim}

However, if you try to combine objects of different types using
\texttt{c()}, R will force (i.e.~coerce) them all to the same data type
(in this case, characters):

\begin{Shaded}
\begin{Highlighting}[]
\NormalTok{string }\OtherTok{\textless{}{-}} \FunctionTok{c}\NormalTok{(}\DecValTok{1}\NormalTok{,}\DecValTok{2}\NormalTok{,}\DecValTok{3}\NormalTok{,}\DecValTok{4}\NormalTok{,}\DecValTok{5}\NormalTok{,}\StringTok{"six"}\NormalTok{) }\CommentTok{\# Create an object called \textquotesingle{}string\textquotesingle{} containing the numbers 1{-}5 and the word "six"}
\NormalTok{string }\CommentTok{\# View the contents of string}
\end{Highlighting}
\end{Shaded}

\begin{verbatim}
[1] "1"   "2"   "3"   "4"   "5"   "six"
\end{verbatim}

\begin{Shaded}
\begin{Highlighting}[]
\FunctionTok{typeof}\NormalTok{(string) }\CommentTok{\# Check the data type of string}
\end{Highlighting}
\end{Shaded}

\begin{verbatim}
[1] "character"
\end{verbatim}

\begin{Shaded}
\begin{Highlighting}[]
\FunctionTok{is.atomic}\NormalTok{(string) }\CommentTok{\# Check if string is an atomic vector}
\end{Highlighting}
\end{Shaded}

\begin{verbatim}
[1] TRUE
\end{verbatim}

To ensure that the original data type is preserved, you can instead use
a list:

\begin{Shaded}
\begin{Highlighting}[]
\NormalTok{string }\OtherTok{\textless{}{-}} \FunctionTok{list}\NormalTok{(}\DecValTok{1}\NormalTok{,}\DecValTok{2}\NormalTok{,}\DecValTok{3}\NormalTok{,}\DecValTok{4}\NormalTok{,}\DecValTok{5}\NormalTok{,}\StringTok{"six"}\NormalTok{) }\CommentTok{\# Create an object called \textquotesingle{}string\textquotesingle{} containing the numbers 1{-}5 and the word "six". stored as a list}
\FunctionTok{typeof}\NormalTok{(string) }\CommentTok{\# Check the data type of string}
\end{Highlighting}
\end{Shaded}

\begin{verbatim}
[1] "list"
\end{verbatim}

\begin{Shaded}
\begin{Highlighting}[]
\FunctionTok{is.atomic}\NormalTok{(string) }\CommentTok{\# Check if string is an atomic vector}
\end{Highlighting}
\end{Shaded}

\begin{verbatim}
[1] FALSE
\end{verbatim}

You can even confirm that the data type of each element in the list has
been preserved (this code uses a process called subsetting, a topic that
will be discussed later):

\begin{Shaded}
\begin{Highlighting}[]
\CommentTok{\# Determine the data type of the 2nd element in the list}
\FunctionTok{typeof}\NormalTok{(string[[}\DecValTok{2}\NormalTok{]]) }
\end{Highlighting}
\end{Shaded}

\begin{verbatim}
[1] "double"
\end{verbatim}

\begin{Shaded}
\begin{Highlighting}[]
\CommentTok{\# Determine the data type of the 6th element in the list}
\FunctionTok{typeof}\NormalTok{(string[[}\DecValTok{6}\NormalTok{]])}
\end{Highlighting}
\end{Shaded}

\begin{verbatim}
[1] "character"
\end{verbatim}

While atomic vectors and lists are one-dimensional data structures, data
can also be stored in multiple dimensions. A two-dimensional structure
that contains data of the same type is called a matrix. There are
multiple ways to create a matrix. One of them is by using a column bind
(\texttt{cbind()}) or row bind (\texttt{rbind()}):

\begin{Shaded}
\begin{Highlighting}[]
\NormalTok{matrix\_A }\OtherTok{\textless{}{-}} \FunctionTok{cbind}\NormalTok{(}\FunctionTok{c}\NormalTok{(}\DecValTok{1}\NormalTok{,}\DecValTok{2}\NormalTok{,}\DecValTok{3}\NormalTok{),}\FunctionTok{c}\NormalTok{(}\DecValTok{4}\NormalTok{,}\DecValTok{5}\NormalTok{,}\DecValTok{6}\NormalTok{)) }\CommentTok{\# Create a matrix using cbind and store it as an object called \textquotesingle{}matrix\_A\textquotesingle{}}
\NormalTok{matrix\_A }\CommentTok{\# View the contents of matrix\_A}
\end{Highlighting}
\end{Shaded}

\begin{verbatim}
     [,1] [,2]
[1,]    1    4
[2,]    2    5
[3,]    3    6
\end{verbatim}

\begin{Shaded}
\begin{Highlighting}[]
\FunctionTok{typeof}\NormalTok{(matrix\_A) }\CommentTok{\# Check the data type of matrix\_A}
\end{Highlighting}
\end{Shaded}

\begin{verbatim}
[1] "double"
\end{verbatim}

\begin{Shaded}
\begin{Highlighting}[]
\FunctionTok{is.matrix}\NormalTok{(matrix\_A) }\CommentTok{\# Check if matrix\_A is a matrix}
\end{Highlighting}
\end{Shaded}

\begin{verbatim}
[1] TRUE
\end{verbatim}

\begin{Shaded}
\begin{Highlighting}[]
\NormalTok{matrix\_B }\OtherTok{\textless{}{-}} \FunctionTok{rbind}\NormalTok{(}\FunctionTok{c}\NormalTok{(}\DecValTok{1}\NormalTok{,}\DecValTok{2}\NormalTok{,}\DecValTok{3}\NormalTok{),}\FunctionTok{c}\NormalTok{(}\DecValTok{4}\NormalTok{,}\DecValTok{5}\NormalTok{,}\DecValTok{6}\NormalTok{)) }\CommentTok{\# Create a matrix using rbind and store it as an object called \textquotesingle{}matrix\_B\textquotesingle{}}
\NormalTok{matrix\_B }\CommentTok{\# View the contents of matrix\_B}
\end{Highlighting}
\end{Shaded}

\begin{verbatim}
     [,1] [,2] [,3]
[1,]    1    2    3
[2,]    4    5    6
\end{verbatim}

\begin{Shaded}
\begin{Highlighting}[]
\FunctionTok{typeof}\NormalTok{(matrix\_B) }\CommentTok{\# Check the data type of matrix\_B}
\end{Highlighting}
\end{Shaded}

\begin{verbatim}
[1] "double"
\end{verbatim}

\begin{Shaded}
\begin{Highlighting}[]
\FunctionTok{is.matrix}\NormalTok{(matrix\_B) }\CommentTok{\# Check if matrix\_B is a matrix}
\end{Highlighting}
\end{Shaded}

\begin{verbatim}
[1] TRUE
\end{verbatim}

As with using c() above to create a vector, R will coerce everything to
the same data type:

\begin{Shaded}
\begin{Highlighting}[]
\NormalTok{matrix\_C }\OtherTok{\textless{}{-}} \FunctionTok{rbind}\NormalTok{(}\FunctionTok{c}\NormalTok{(}\DecValTok{1}\NormalTok{,}\DecValTok{2}\NormalTok{,}\StringTok{"three"}\NormalTok{),}\FunctionTok{c}\NormalTok{(}\DecValTok{4}\NormalTok{,}\DecValTok{5}\NormalTok{,}\DecValTok{6}\NormalTok{)) }\CommentTok{\# Create a matrix using rbind and store it as an object called \textquotesingle{}matrix\_C\textquotesingle{}}
\NormalTok{matrix\_C }\CommentTok{\# View the contents of matrix\_C}
\end{Highlighting}
\end{Shaded}

\begin{verbatim}
     [,1] [,2] [,3]   
[1,] "1"  "2"  "three"
[2,] "4"  "5"  "6"    
\end{verbatim}

\begin{Shaded}
\begin{Highlighting}[]
\FunctionTok{typeof}\NormalTok{(matrix\_C) }\CommentTok{\# Check the data type of matrix\_C}
\end{Highlighting}
\end{Shaded}

\begin{verbatim}
[1] "character"
\end{verbatim}

Even if you try to make one row of the matrix a list to preserve the
different data types, R will coerce every row of the matrix to a list so
that it is technically still the same data type throughout:

\begin{Shaded}
\begin{Highlighting}[]
\NormalTok{matrix\_C }\OtherTok{\textless{}{-}} \FunctionTok{rbind}\NormalTok{(}\FunctionTok{list}\NormalTok{(}\DecValTok{1}\NormalTok{,}\DecValTok{2}\NormalTok{,}\StringTok{"three"}\NormalTok{),}\FunctionTok{c}\NormalTok{(}\DecValTok{4}\NormalTok{,}\DecValTok{5}\NormalTok{,}\DecValTok{6}\NormalTok{)) }\CommentTok{\# Create a matrix using rbind and store it as an object called \textquotesingle{}matrix\_C\textquotesingle{}}
\NormalTok{matrix\_C }\CommentTok{\# View the contents of matrix\_C}
\end{Highlighting}
\end{Shaded}

\begin{verbatim}
     [,1] [,2] [,3]   
[1,] 1    2    "three"
[2,] 4    5    6      
\end{verbatim}

\begin{Shaded}
\begin{Highlighting}[]
\FunctionTok{typeof}\NormalTok{(matrix\_C) }\CommentTok{\# Check the data type of matrix\_C}
\end{Highlighting}
\end{Shaded}

\begin{verbatim}
[1] "list"
\end{verbatim}

\begin{Shaded}
\begin{Highlighting}[]
\FunctionTok{is.matrix}\NormalTok{(matrix\_C) }\CommentTok{\# Check if matrix\_C is a matrix}
\end{Highlighting}
\end{Shaded}

\begin{verbatim}
[1] TRUE
\end{verbatim}

\begin{Shaded}
\begin{Highlighting}[]
\FunctionTok{typeof}\NormalTok{(matrix\_C[}\DecValTok{2}\NormalTok{,]) }\CommentTok{\# Check the type of data stored in row 2 of matrix\_C}
\end{Highlighting}
\end{Shaded}

\begin{verbatim}
[1] "list"
\end{verbatim}

So while this technically does allow the matrix to hold different data
types, storing them within lists prevents certain matrix operations from
being able to be used on the matrix.

To preserve different data types in a two-dimensional structure without
needing to store them as lists, you can instead use a data frame. This
will allow atomic vectors of different types to be combined as columns
within a single structure:

\begin{Shaded}
\begin{Highlighting}[]
\NormalTok{dataframe\_A }\OtherTok{\textless{}{-}} \FunctionTok{data.frame}\NormalTok{(}\FunctionTok{c}\NormalTok{(}\DecValTok{1}\NormalTok{,}\DecValTok{2}\NormalTok{,}\DecValTok{3}\NormalTok{),}\FunctionTok{c}\NormalTok{(}\StringTok{"four"}\NormalTok{, }\StringTok{"five"}\NormalTok{, }\StringTok{"six"}\NormalTok{)) }\CommentTok{\# Create a data frame and store it as an object called dataframe\_A}
\NormalTok{dataframe\_A }\CommentTok{\# View dataframe\_A}
\end{Highlighting}
\end{Shaded}

\begin{verbatim}
  c.1..2..3. c..four....five....six..
1          1                     four
2          2                     five
3          3                      six
\end{verbatim}

\begin{Shaded}
\begin{Highlighting}[]
\FunctionTok{typeof}\NormalTok{(dataframe\_A[,}\DecValTok{1}\NormalTok{]) }\CommentTok{\# Check type of data in column 1 of dataframe\_A}
\end{Highlighting}
\end{Shaded}

\begin{verbatim}
[1] "double"
\end{verbatim}

\begin{Shaded}
\begin{Highlighting}[]
\FunctionTok{typeof}\NormalTok{(dataframe\_A[,}\DecValTok{2}\NormalTok{]) }\CommentTok{\# Check type of data in column 2 of dataframe\_A}
\end{Highlighting}
\end{Shaded}

\begin{verbatim}
[1] "character"
\end{verbatim}

\begin{Shaded}
\begin{Highlighting}[]
\FunctionTok{is.data.frame}\NormalTok{(dataframe\_A) }\CommentTok{\# Check if dataframe\_A is a data frame?}
\end{Highlighting}
\end{Shaded}

\begin{verbatim}
[1] TRUE
\end{verbatim}

Finally, a data structure with two or more dimensions is called an
array. Arrays are similar to matrices in that they can only store one
data type (unless you store the data as lists as described above for
matrices, which limits which array functions can be used):

\begin{Shaded}
\begin{Highlighting}[]
\NormalTok{array\_A }\OtherTok{\textless{}{-}} \FunctionTok{array}\NormalTok{(}\DecValTok{1}\SpecialCharTok{:}\DecValTok{12}\NormalTok{, }\AttributeTok{dim =} \FunctionTok{c}\NormalTok{(}\DecValTok{2}\NormalTok{, }\DecValTok{3}\NormalTok{, }\DecValTok{2}\NormalTok{)) }\CommentTok{\# Create an array using the numbers 1{-}12 stored in 2 rows, 3 columns, and 2 "layers" (i.e. the third dimension) and store it as an object called array\_A}
\NormalTok{array\_A }\CommentTok{\# View the contents of array\_A}
\end{Highlighting}
\end{Shaded}

\begin{verbatim}
, , 1

     [,1] [,2] [,3]
[1,]    1    3    5
[2,]    2    4    6

, , 2

     [,1] [,2] [,3]
[1,]    7    9   11
[2,]    8   10   12
\end{verbatim}

The most common data structure used in bioinformatics is a data frame,
though you may also use matrices for simpler datasets. Knowing how each
data structure is composed and what data type(s) can be stored within
them will be very useful when trying to perform operations on these
structures (as will be discussed later).




\end{document}
